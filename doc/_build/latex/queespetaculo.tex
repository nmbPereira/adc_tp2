%% Generated by Sphinx.
\def\sphinxdocclass{report}
\documentclass[letterpaper,10pt,portuges]{sphinxmanual}
\ifdefined\pdfpxdimen
   \let\sphinxpxdimen\pdfpxdimen\else\newdimen\sphinxpxdimen
\fi \sphinxpxdimen=.75bp\relax
\ifdefined\pdfimageresolution
    \pdfimageresolution= \numexpr \dimexpr1in\relax/\sphinxpxdimen\relax
\fi
%% let collapsible pdf bookmarks panel have high depth per default
\PassOptionsToPackage{bookmarksdepth=5}{hyperref}


\PassOptionsToPackage{warn}{textcomp}
\usepackage[utf8]{inputenc}
\ifdefined\DeclareUnicodeCharacter
% support both utf8 and utf8x syntaxes
  \ifdefined\DeclareUnicodeCharacterAsOptional
    \def\sphinxDUC#1{\DeclareUnicodeCharacter{"#1}}
  \else
    \let\sphinxDUC\DeclareUnicodeCharacter
  \fi
  \sphinxDUC{00A0}{\nobreakspace}
  \sphinxDUC{2500}{\sphinxunichar{2500}}
  \sphinxDUC{2502}{\sphinxunichar{2502}}
  \sphinxDUC{2514}{\sphinxunichar{2514}}
  \sphinxDUC{251C}{\sphinxunichar{251C}}
  \sphinxDUC{2572}{\textbackslash}
\fi
\usepackage{cmap}
\usepackage[T1]{fontenc}
\usepackage{amsmath,amssymb,amstext}
\usepackage{babel}



\usepackage{tgtermes}
\usepackage{tgheros}
\renewcommand{\ttdefault}{txtt}



\usepackage[Sonny]{fncychap}
\ChNameVar{\Large\normalfont\sffamily}
\ChTitleVar{\Large\normalfont\sffamily}
\usepackage{sphinx}

\fvset{fontsize=auto}
\usepackage{geometry}


% Include hyperref last.
\usepackage{hyperref}
% Fix anchor placement for figures with captions.
\usepackage{hypcap}% it must be loaded after hyperref.
% Set up styles of URL: it should be placed after hyperref.
\urlstyle{same}

\addto\captionsportuges{\renewcommand{\contentsname}{Contents:}}

\usepackage{sphinxmessages}
\setcounter{tocdepth}{1}



\title{Que Espetaculo}
\date{19 dez., 2022}
\release{0.1}
\author{Nuno Pereira, Daniel Fordham, Lourenço Albuquerque}
\newcommand{\sphinxlogo}{\vbox{}}
\renewcommand{\releasename}{Versão}
\makeindex
\begin{document}

\ifdefined\shorthandoff
  \ifnum\catcode`\=\string=\active\shorthandoff{=}\fi
  \ifnum\catcode`\"=\active\shorthandoff{"}\fi
\fi

\pagestyle{empty}
\sphinxmaketitle
\pagestyle{plain}
\sphinxtableofcontents
\pagestyle{normal}
\phantomsection\label{\detokenize{index::doc}}


\sphinxstepscope


\chapter{src}
\label{\detokenize{modules:src}}\label{\detokenize{modules::doc}}
\sphinxstepscope


\section{espetaculos module}
\label{\detokenize{espetaculos:module-espetaculos}}\label{\detokenize{espetaculos:espetaculos-module}}\label{\detokenize{espetaculos::doc}}\index{módulo@\spxentry{módulo}!espetaculos@\spxentry{espetaculos}}\index{espetaculos@\spxentry{espetaculos}!módulo@\spxentry{módulo}}
\sphinxAtStartPar
Funções de espetáculos
\begin{quote}\begin{description}
\sphinxlineitem{Date}
\sphinxAtStartPar
6 Dez 2022

\sphinxlineitem{Version}
\sphinxAtStartPar
0.1

\sphinxlineitem{Authors}
\sphinxAtStartPar
Nuno Pereira, Daniel Fordham, Lourenço Albuquerque

\end{description}\end{quote}
\index{criar\_espetaculo() (no módulo espetaculos)@\spxentry{criar\_espetaculo()}\spxextra{no módulo espetaculos}}

\begin{fulllineitems}
\phantomsection\label{\detokenize{espetaculos:espetaculos.criar_espetaculo}}
\pysigstartsignatures
\pysiglinewithargsret{\sphinxcode{\sphinxupquote{espetaculos.}}\sphinxbfcode{\sphinxupquote{criar\_espetaculo}}}{}{}
\pysigstopsignatures
\sphinxAtStartPar
Cria um novo espetáculo a partir de inputs do utilizador
\begin{quote}\begin{description}
\sphinxlineitem{Retorno}
\sphinxAtStartPar
dicionario de um espetáculo na forma
«nome»: \textless{}\textless{}nome\textgreater{}\textgreater{}, «artista»: \textless{}\textless{}artista\textgreater{}\textgreater{}, «local»: \textless{}\textless{}local\textgreater{}\textgreater{}

\end{description}\end{quote}

\end{fulllineitems}

\index{imprime\_lista\_espetaculos() (no módulo espetaculos)@\spxentry{imprime\_lista\_espetaculos()}\spxextra{no módulo espetaculos}}

\begin{fulllineitems}
\phantomsection\label{\detokenize{espetaculos:espetaculos.imprime_lista_espetaculos}}
\pysigstartsignatures
\pysiglinewithargsret{\sphinxcode{\sphinxupquote{espetaculos.}}\sphinxbfcode{\sphinxupquote{imprime\_lista\_espetaculos}}}{\emph{\DUrole{n}{lista\_de\_espetaculos}}}{}
\pysigstopsignatures
\sphinxAtStartPar
Imprime a lista de espetáculos.
\begin{quote}\begin{description}
\sphinxlineitem{Parâmetros}
\sphinxAtStartPar
\sphinxstyleliteralstrong{\sphinxupquote{lista\_de\_espetaculos}} \textendash{} 

\sphinxlineitem{Retorno}
\sphinxAtStartPar


\end{description}\end{quote}

\end{fulllineitems}


\sphinxstepscope


\section{io\_ficheiros module}
\label{\detokenize{io_ficheiros:module-io_ficheiros}}\label{\detokenize{io_ficheiros:io-ficheiros-module}}\label{\detokenize{io_ficheiros::doc}}\index{módulo@\spxentry{módulo}!io\_ficheiros@\spxentry{io\_ficheiros}}\index{io\_ficheiros@\spxentry{io\_ficheiros}!módulo@\spxentry{módulo}}
\sphinxAtStartPar
Funções de i/o de ficheiros
\begin{quote}\begin{description}
\sphinxlineitem{Date}
\sphinxAtStartPar
6 Dez 2022

\sphinxlineitem{Version}
\sphinxAtStartPar
0.1

\sphinxlineitem{Authors}
\sphinxAtStartPar
Nuno Pereira, Daniel Fordham, Lourenço Albuquerque

\end{description}\end{quote}
\index{guardar\_ficheiro() (no módulo io\_ficheiros)@\spxentry{guardar\_ficheiro()}\spxextra{no módulo io\_ficheiros}}

\begin{fulllineitems}
\phantomsection\label{\detokenize{io_ficheiros:io_ficheiros.guardar_ficheiro}}
\pysigstartsignatures
\pysiglinewithargsret{\sphinxcode{\sphinxupquote{io\_ficheiros.}}\sphinxbfcode{\sphinxupquote{guardar\_ficheiro}}}{\emph{\DUrole{n}{ficheiro}}, \emph{\DUrole{n}{dados}}}{}
\pysigstopsignatures
\sphinxAtStartPar
Guardar os dados num ficheiro
\begin{quote}\begin{description}
\sphinxlineitem{Parâmetros}\begin{itemize}
\item {} 
\sphinxAtStartPar
\sphinxstyleliteralstrong{\sphinxupquote{ficheiro}} \textendash{} nome do ficheiro onde vai guardar os dados

\item {} 
\sphinxAtStartPar
\sphinxstyleliteralstrong{\sphinxupquote{dados}} \textendash{} dados para guardar

\end{itemize}

\end{description}\end{quote}

\end{fulllineitems}

\index{ler\_ficheiro() (no módulo io\_ficheiros)@\spxentry{ler\_ficheiro()}\spxextra{no módulo io\_ficheiros}}

\begin{fulllineitems}
\phantomsection\label{\detokenize{io_ficheiros:io_ficheiros.ler_ficheiro}}
\pysigstartsignatures
\pysiglinewithargsret{\sphinxcode{\sphinxupquote{io\_ficheiros.}}\sphinxbfcode{\sphinxupquote{ler\_ficheiro}}}{\emph{\DUrole{n}{ficheiro}}}{}
\pysigstopsignatures
\sphinxAtStartPar
Lê os dados de um ficheiro
\begin{quote}\begin{description}
\sphinxlineitem{Parâmetros}
\sphinxAtStartPar
\sphinxstyleliteralstrong{\sphinxupquote{ficheiro}} \textendash{} nome do ficheiro para ler

\sphinxlineitem{Retorno}
\sphinxAtStartPar
o conteúdo do ficheiro (depende dos dados guardados)

\end{description}\end{quote}

\end{fulllineitems}


\sphinxstepscope


\section{io\_terminal module}
\label{\detokenize{io_terminal:module-io_terminal}}\label{\detokenize{io_terminal:io-terminal-module}}\label{\detokenize{io_terminal::doc}}\index{módulo@\spxentry{módulo}!io\_terminal@\spxentry{io\_terminal}}\index{io\_terminal@\spxentry{io\_terminal}!módulo@\spxentry{módulo}}
\sphinxAtStartPar
Funções de i/o de terminal
\begin{quote}\begin{description}
\sphinxlineitem{Date}
\sphinxAtStartPar
6 Dez 2022

\sphinxlineitem{Version}
\sphinxAtStartPar
0.1

\sphinxlineitem{Authors}
\sphinxAtStartPar
Nuno Pereira, Daniel Fordham, Lourenço Albuquerque

\end{description}\end{quote}
\index{imprime\_lista() (no módulo io\_terminal)@\spxentry{imprime\_lista()}\spxextra{no módulo io\_terminal}}

\begin{fulllineitems}
\phantomsection\label{\detokenize{io_terminal:io_terminal.imprime_lista}}
\pysigstartsignatures
\pysiglinewithargsret{\sphinxcode{\sphinxupquote{io\_terminal.}}\sphinxbfcode{\sphinxupquote{imprime\_lista}}}{\emph{\DUrole{n}{cabecalho}}, \emph{\DUrole{n}{lista}}}{}
\pysigstopsignatures
\sphinxAtStartPar
Imprime a \sphinxcode{\sphinxupquote{lista}} na forma de uma tabela com um cabeçalho

\sphinxAtStartPar
Recebe uma lista na forma {[}\{«atrib1»: valor 1, «atrib2»: valor 2, …\},
\{«atrib1»: valor 1, «atrib2»: valor 2, …\}, …{]} e imprime no terminal uma tabela
na forma


\begin{savenotes}\sphinxattablestart
\sphinxthistablewithglobalstyle
\centering
\begin{tabulary}{\linewidth}[t]{|T|T|T|}
\sphinxtoprule
\sphinxstyletheadfamily 
\sphinxAtStartPar
id
&\sphinxstyletheadfamily 
\sphinxAtStartPar
atrib1
&\sphinxstyletheadfamily 
\sphinxAtStartPar
atrib2
\\
\sphinxmidrule
\sphinxtableatstartofbodyhook
\sphinxAtStartPar
0
&
\sphinxAtStartPar
valor1
&
\sphinxAtStartPar
valor2
\\
\sphinxhline
\sphinxAtStartPar
1
&
\sphinxAtStartPar
…
&
\sphinxAtStartPar
…
\\
\sphinxbottomrule
\end{tabulary}
\sphinxtableafterendhook\par
\sphinxattableend\end{savenotes}
\begin{quote}\begin{description}
\sphinxlineitem{Parâmetros}\begin{itemize}
\item {} 
\sphinxAtStartPar
\sphinxstyleliteralstrong{\sphinxupquote{cabecalho}} \textendash{} cabecalho para a listagem

\item {} 
\sphinxAtStartPar
\sphinxstyleliteralstrong{\sphinxupquote{lista}} \textendash{} lista a ser impressa

\end{itemize}

\end{description}\end{quote}

\end{fulllineitems}

\index{imprime\_lista\_de\_dicionarios() (no módulo io\_terminal)@\spxentry{imprime\_lista\_de\_dicionarios()}\spxextra{no módulo io\_terminal}}

\begin{fulllineitems}
\phantomsection\label{\detokenize{io_terminal:io_terminal.imprime_lista_de_dicionarios}}
\pysigstartsignatures
\pysiglinewithargsret{\sphinxcode{\sphinxupquote{io\_terminal.}}\sphinxbfcode{\sphinxupquote{imprime\_lista\_de\_dicionarios}}}{\emph{\DUrole{n}{lista}}}{}
\pysigstopsignatures
\sphinxAtStartPar
…. todo ….
:param lista:
:return:

\end{fulllineitems}


\sphinxstepscope


\section{main module}
\label{\detokenize{main:module-main}}\label{\detokenize{main:main-module}}\label{\detokenize{main::doc}}\index{módulo@\spxentry{módulo}!main@\spxentry{main}}\index{main@\spxentry{main}!módulo@\spxentry{módulo}}
\sphinxAtStartPar
Menu principal da aplicação de venda de bilhetes
\begin{quote}\begin{description}
\sphinxlineitem{Date}
\sphinxAtStartPar
6 Dez 2022

\sphinxlineitem{Version}
\sphinxAtStartPar
0.1

\sphinxlineitem{Authors}
\sphinxAtStartPar
Nuno Pereira, Daniel Fordham, Lourenço Albuquerque

\end{description}\end{quote}
\index{carregar\_dados() (no módulo main)@\spxentry{carregar\_dados()}\spxextra{no módulo main}}

\begin{fulllineitems}
\phantomsection\label{\detokenize{main:main.carregar_dados}}
\pysigstartsignatures
\pysiglinewithargsret{\sphinxcode{\sphinxupquote{main.}}\sphinxbfcode{\sphinxupquote{carregar\_dados}}}{}{}
\pysigstopsignatures
\sphinxAtStartPar
Carrega os dados de espetáculos e utilizadores a partir de um ficheiro local
\begin{quote}\begin{description}
\sphinxlineitem{Retorno}
\sphinxAtStartPar
o conteúdo de ambos ficheiro (depende dos dados guardados)

\end{description}\end{quote}

\end{fulllineitems}

\index{guardar\_dados() (no módulo main)@\spxentry{guardar\_dados()}\spxextra{no módulo main}}

\begin{fulllineitems}
\phantomsection\label{\detokenize{main:main.guardar_dados}}
\pysigstartsignatures
\pysiglinewithargsret{\sphinxcode{\sphinxupquote{main.}}\sphinxbfcode{\sphinxupquote{guardar\_dados}}}{\emph{\DUrole{n}{lista\_de\_espetaculos}}, \emph{\DUrole{n}{lista\_de\_utilizadores}}}{}
\pysigstopsignatures
\sphinxAtStartPar
Guarda os dados da sessão para carregar numa sessão futura.
\begin{quote}\begin{description}
\sphinxlineitem{Parâmetros}\begin{itemize}
\item {} 
\sphinxAtStartPar
\sphinxstyleliteralstrong{\sphinxupquote{lista\_de\_espetaculos}} \textendash{} lista de espetáculos da sessão

\item {} 
\sphinxAtStartPar
\sphinxstyleliteralstrong{\sphinxupquote{lista\_de\_utilizadores}} \textendash{} lista de utilizadores da sessão

\end{itemize}

\end{description}\end{quote}

\end{fulllineitems}

\index{menu() (no módulo main)@\spxentry{menu()}\spxextra{no módulo main}}

\begin{fulllineitems}
\phantomsection\label{\detokenize{main:main.menu}}
\pysigstartsignatures
\pysiglinewithargsret{\sphinxcode{\sphinxupquote{main.}}\sphinxbfcode{\sphinxupquote{menu}}}{}{}
\pysigstopsignatures
\sphinxAtStartPar
Menu principal da aplicação

\end{fulllineitems}


\sphinxstepscope


\section{utilizadores module}
\label{\detokenize{utilizadores:module-utilizadores}}\label{\detokenize{utilizadores:utilizadores-module}}\label{\detokenize{utilizadores::doc}}\index{módulo@\spxentry{módulo}!utilizadores@\spxentry{utilizadores}}\index{utilizadores@\spxentry{utilizadores}!módulo@\spxentry{módulo}}
\sphinxAtStartPar
Funções de utilizadores
\begin{quote}\begin{description}
\sphinxlineitem{Date}
\sphinxAtStartPar
6 Dez 2022

\sphinxlineitem{Version}
\sphinxAtStartPar
0.1

\sphinxlineitem{Authors}
\sphinxAtStartPar
Nuno Pereira, Daniel Fordham, Lourenço Albuquerque

\end{description}\end{quote}
\index{cria\_novo\_utilizador() (no módulo utilizadores)@\spxentry{cria\_novo\_utilizador()}\spxextra{no módulo utilizadores}}

\begin{fulllineitems}
\phantomsection\label{\detokenize{utilizadores:utilizadores.cria_novo_utilizador}}
\pysigstartsignatures
\pysiglinewithargsret{\sphinxcode{\sphinxupquote{utilizadores.}}\sphinxbfcode{\sphinxupquote{cria\_novo\_utilizador}}}{}{}
\pysigstopsignatures
\sphinxAtStartPar
Cria um novo utilizador a partir de inputs
\begin{quote}\begin{description}
\sphinxlineitem{Retorno}
\sphinxAtStartPar
dicionario de um utilizador na forma
«nome»: \textless{}\textless{}nome\textgreater{}\textgreater{}, «email»: \textless{}\textless{}email\textgreater{}\textgreater{}

\end{description}\end{quote}

\end{fulllineitems}

\index{imprime\_lista\_de\_utilizadores() (no módulo utilizadores)@\spxentry{imprime\_lista\_de\_utilizadores()}\spxextra{no módulo utilizadores}}

\begin{fulllineitems}
\phantomsection\label{\detokenize{utilizadores:utilizadores.imprime_lista_de_utilizadores}}
\pysigstartsignatures
\pysiglinewithargsret{\sphinxcode{\sphinxupquote{utilizadores.}}\sphinxbfcode{\sphinxupquote{imprime\_lista\_de\_utilizadores}}}{\emph{\DUrole{n}{lista\_de\_utilizadores}}}{}
\pysigstopsignatures
\sphinxAtStartPar
…

\end{fulllineitems}



\chapter{Indices and tables}
\label{\detokenize{index:indices-and-tables}}\begin{itemize}
\item {} 
\sphinxAtStartPar
\DUrole{xref,std,std-ref}{genindex}

\item {} 
\sphinxAtStartPar
\DUrole{xref,std,std-ref}{modindex}

\item {} 
\sphinxAtStartPar
\DUrole{xref,std,std-ref}{search}

\end{itemize}


\renewcommand{\indexname}{Índice de Módulos do Python}
\begin{sphinxtheindex}
\let\bigletter\sphinxstyleindexlettergroup
\bigletter{e}
\item\relax\sphinxstyleindexentry{espetaculos}\sphinxstyleindexpageref{espetaculos:\detokenize{module-espetaculos}}
\indexspace
\bigletter{i}
\item\relax\sphinxstyleindexentry{io\_ficheiros}\sphinxstyleindexpageref{io_ficheiros:\detokenize{module-io_ficheiros}}
\item\relax\sphinxstyleindexentry{io\_terminal}\sphinxstyleindexpageref{io_terminal:\detokenize{module-io_terminal}}
\indexspace
\bigletter{m}
\item\relax\sphinxstyleindexentry{main}\sphinxstyleindexpageref{main:\detokenize{module-main}}
\indexspace
\bigletter{u}
\item\relax\sphinxstyleindexentry{utilizadores}\sphinxstyleindexpageref{utilizadores:\detokenize{module-utilizadores}}
\end{sphinxtheindex}

\renewcommand{\indexname}{Índice}
\printindex
\end{document}